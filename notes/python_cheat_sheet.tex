\documentclass{article}
\usepackage[utf8]{inputenc}
\usepackage{amsmath,amssymb,bbm}
\usepackage{tikz}
\usepackage{multirow, tabularx}

\DeclareMathOperator*{\argmax}{arg\,max}
\DeclareMathOperator*{\argmin}{arg\,min}
\DeclareMathOperator*{\E}{\mathop{\mathbb{E}}}

\newcommand{\red}[1]{\textcolor{red}{#1}}
\newcommand{\redb}[1]{\textcolor{red}{\textbf{#1}}}
\newcommand{\blue}[1]{\textcolor{blue}{#1}}
\newcommand{\newpara}{\leavevmode\newline}
\newcommand{\hrfullline}{\noindent\makebox[\linewidth]{\rule{\paperwidth}{2pt}}}
\newcommand{\mcaly}{\mathcal{Y}}
\newcommand{\xhat}{\hat{x}}

\usepackage{hyperref}
\hypersetup{
    colorlinks=true,
    linkcolor=blue,
    filecolor=magenta,      
    urlcolor=blue,
}

\title{Python Cheat Sheet}
\author{Girish Kumar }
\date{Oct 2020}

\begin{document}
\maketitle

\setcounter{secnumdepth}{0}
\tableofcontents

\newpage
\section{Data Structures}

\subsection{Set}
\begin{itemize}
    \item s = set(), unordered, immutable
    \item item in s = boolean
    \item s.add(item), s.update(item\_list), s.remove(item), s.clear()
\end{itemize}

\subsection{Heap}
\begin{itemize}
    \item import heapq
    \item only min heap implementation
    \item for max heap multiply item by (-1) during push and pop
    \item initialize a list  as q = []
    \item heappush(q, item)
    \item item = heappop(q)
    \item nlargest, nsmallest(n, q), heapify, merge
\end{itemize}

\hrfullline

\end{document}
