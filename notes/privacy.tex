\documentclass{article}
%! Author = girish
%! Date = 5/7/21


% Packages
\usepackage{amsmath, amssymb, bbm}


% -------------------------------------------
% Math operators

\DeclareMathOperator*{\argmax}{arg\,max}
\DeclareMathOperator*{\argmin}{arg\,min}

\newcommand{\mcaly}{\mathcal{Y}}
\newcommand{\xhat}{\hat{x}}

\DeclareMathOperator*{\E}{\mathop{\mathbb{E}}}


% -------------------------------------------
% Formatting

\newcommand{\red}[1]{\textcolor{red}{#1}}
\newcommand{\blue}[1]{\textcolor{blue}{#1}}
\newcommand{\newpara}{\leavevmode\newline}
\newcommand{\hrfullline}{\noindent\makebox[\linewidth]{\rule{\paperwidth}{2pt}}}


\usepackage[utf8]{inputenc}
\usepackage{tikz}
\usepackage{multirow, tabularx}

\usepackage{hyperref}
\hypersetup{
    colorlinks=true,
    linkcolor=blue,
    filecolor=magenta,
    urlcolor=blue,
}

\title{Summary of some important concepts}
\author{Girish Kumar }
\date{Oct 2020}

\begin{document}
\maketitle

\tableofcontents
\newpage

\section{Differntial Privacy}

\begin{itemize}
    \item Some example problems
        \begin{itemize}
            \item \textbf{Linkage Attack}: Re-identification of an individual example by using an auxiliary data-set to identify key personal information (Netflix challenge and IMDB)
            \item Finding Connections by attacking on an anonymous graph
            \item \textbf{Reconstruction Attacks}: Identification (or ruling out) of  some characteristics
        \end{itemize}
    \item Some not so good suggestions for privacy
        \begin{itemize}
            \item Only allow large set queries - fails to a differencing attack such as (Query details of a set $\mathbb{S}$ - Query details of a set $\mathbb{S}-\{x\}$)
            \item Add random noise to the result - fails as expectation of repeated queries gives the true result
            \item Detect when answering is unsafe - refusal can be disclosive
            \item Only publish few complete results of individuals who are ok with public data - works but is that what we really want?
        \end{itemize}
    \item \textbf{Definition}: M gives $\epsilon$ differential privacy if for all pairs of databases $x, x^\prime$ differing in one row, and all subsets of C of possible outputs,
    $$Pr[M(x) \in C] \leq e^\epsilon Pr[M(x^\prime) \in C]$$
    or equivalently,
    $$ln\left( \frac{Pr[M(x) \in C]}{Pr[M(x^\prime) \in C]} \right) \leq \epsilon$$
    \item \textbf{Why this algorithm is better?}
        \begin{itemize}
            \item Is not just focused on re-identification
            \item Is the same even if some Auxiliary information is available
            \item Quantifiable privacy loss (also allows comparison of techniques)
            \item Allows unconditional composition
                $$ln\left( \frac{Pr[M_{1,2}(x) = (r_1, r_2)]}{Pr[M_{1,2}(x^\prime) = (r_1, r_2)]} \right) = ln\left( \frac{Pr[M_1(x) = r_1]Pr[M_2(x) = r_2]}{Pr[M_1(x^\prime) = r_1]Pr[M_2(x^\prime) = r_2]} \right) \leq \epsilon_1+\epsilon_2$$
        \end{itemize}
    \item \textbf{Some important points}
    \begin{itemize}
        \item Why $e^\epsilon$? - Allows for composition
        \item Why multiplicative bound? - Other definition are not as efficient definition of privacy \blue{Why?}
        \item What does change in one row means when considering neighbouring datasets in the definition? It could mean one of the two things-
            \begin{itemize}
                \item Unbounded DP: One data point is not present in the other dataset.
                \item Bounded DP: One data point is replaced by another one. This is similar to the bounded DP except that the factor $\epsilon$ can be thought of here being equivalent to $2\epsilon$ for the unbounded case.
            \end{itemize}
            Theoretically they are not that different, but implementation wise need to be careful when using $\epsilon$ to calculate other things such as required data-points to achieve that particular budget.
        \item What is a good $\epsilon$? - Typically about $0.1 \leq \epsilon \leq 5$ but depends on the situation.
    \end{itemize}
    \item \textbf{What DP doesn't allow?}
        \begin{itemize}
            \item Although individual statistic are protected, population statistics may not be
            \item Consequently, not for inference of individual statistics
            \item Information theoretical bound and not computational one (such as RSA)
        \end{itemize}
    \item \textbf{Sensitivity of a function}
        $$\Delta f = \max_{x, x^\prime} \Vert f(x)-f(x^\prime) \Vert_1$$
        where $x$ and $x^\prime$ differ only by one row
\end{itemize}

\subsection{Achieve Differential Privacy}
\subsubsection{Randomized Response}
\begin{itemize}
    \item Assume for a boolean data with true value $X$, the curator (or individual in the case of localized DP) will respond with $Y$, such that for some $0 \leq \alpha \leq 1/2$,
    $$Y = \begin{cases} X, & \text{w.p. } \frac{1}{2}+\alpha \\ 1-X, & \text{w.p. } \frac{1}{2}-\alpha \end{cases}$$
    \item $\alpha=0 \Rightarrow$ complete privacy, no utility
    \item $\alpha=1/2 \Rightarrow$ no privacy, complete utility
    \item For a sequence of data points $(X_1, X_2, \dots, X_n)$ this is DP with $\epsilon = \ln{\left(\frac{\frac{1}{2}+\alpha}{\frac{1}{2}-\alpha}\right)} \approx O(\alpha)$
    \item How to measure the accuracy of the algorithm? - One thing we can do is how accurately do we estimate the population mean? We have,
    \begin{align*}
        \expect{Y} &= X(\frac{1}{2} + \alpha) + (1-X) (\frac{1}{2} - \alpha)\\
        &= 2\alpha X + \frac{1}{2} - \alpha
    \end{align*}
    Hence an unbiased estimator for $X$ is $Z = \frac{Y+\alpha - 1/2}{2\alpha}$
    Moreover,
    \begin{align*}
        \var{Z} &= \frac{1}{4\alpha^2} \var{Y} \leq \frac{1}{16 \alpha^2}
    \end{align*}
    So, for a sequence of values $X = (X_1, X_2, \dots, X_n)$ with randomized response $Y = (Y_1, Y_2, \dots, Y_n)$ we have the estimate of expectation as,
    \begin{align*}
        \hat \mu &= \frac{1}{n} \sum_{i=1}^n Z_i\\
        \Rightarrow \var{\hat\mu} &\leq \frac{1}{16\alpha^2 n^2} (n) = \frac{1}{16 \alpha^2 n}
    \end{align*}
    By Chebyshev's inequality,
    $$|\hat\mu - \mu| \leq O(\frac{1}{\alpha \sqrt{n}})$$
    \blue{WHY?}
\end{itemize}


\subsubsection{Laplace Mechanism}

\begin{itemize}
    \item About Laplace distribution
        \begin{itemize}
            \item Laplace distribution pdf with mean $\mu$ and variance $2b^2$ is given by
                $$Laplace(\mu,b) = f(x\mid \mu ,b) ={\frac  {1}{2b}}\exp \left(-{\frac  {|x-\mu |}{b}}\right)$$
                \item Double sided exponential distribution
            \item Tail is fatter than Gaussian and the peak is pointy, key difference is $L_1$ norm instead of $L_2$ norm
        \end{itemize}
    \item To achieve $\epsilon$ differential privacy in the output of a query $f$ of sensitivity $\Delta$, add noise sampled from the distribution $Lap(\Delta / \epsilon)$. Let us call this as the output $M$ of our DP algorithm.
    \item Algorithm is $\epsilon$-DP - Let $X$ and $Y$ denote neighbouring datasets and $f: \mathbb{R}^n \to \mathbb{R}^k$. Let $P_{M(X)}(z)$ denote the probability density for $M(X)=z$, then,
        \begin{align*}
            \frac{P_{M(X)}(z)}{P_{M(Y)}(z)} &= \frac{\prod_{i=1}^k M(X)_i=z_i}{\prod_{i=1}^k M(Y)_i=z_i}\\
            &= \frac{\prod_{i=1}^k \exp{(-\frac{\epsilon |f(X)_i - z_i|}{\Delta}) }}{\prod_{i=1}^k \exp{(-\frac{\epsilon |f(Y)_i - z_i|}{\Delta}) }}
            = \frac{ \exp{( \sum_{i=1}^k -\frac{\epsilon |f(X)_i - z_i|}{\Delta}) }}{ \exp{( \sum_{i=1}^k -\frac{\epsilon |f(Y)_i - z_i|}{\Delta}) }}\\
            &= \exp{\left( \frac{\epsilon}{\Delta} \sum_{i=1}^k \left(  |f(Y)_i - z_i| - |f(X)_i - z_i| \right) \right)}\\
            &\leq \exp{\left( \frac{\epsilon}{\Delta} \sum_{i=1}^k \left(  |f(Y)_i - f(X)_i| \right) \right)}\\
            &\leq \exp{ \left(\frac{\epsilon}{\Delta} \Delta\right)} = e^\epsilon\\
        \end{align*}
    \item Accuracy is of the order $O(\frac{\Delta}{\epsilon})$:
        \begin{align*}
            \expect{M(X)} &= \expect{f(X) + Lap(\Delta/\epsilon)} = f(X)\\
            \var{M(X)} &= \var{Lap(\Delta/ \epsilon)} = \frac{2 \Delta^2}{\epsilon^2}\\
            M(X)-f(X) &\sim Lap(\Delta/\epsilon)\\
            \Rightarrow \prob{|M(X)-f(X)| > k\frac{\Delta}{\epsilon}} &= e^{-k}\\
            \Rightarrow \prob{|M(X)-f(X)| > \ln{(1/k)}\frac{\Delta}{\epsilon}} &= k\\
            \Rightarrow |M(X)-f(X)| &\leq \ln{(1/k)}\frac{\Delta}{\epsilon} \text{ w.p. } 1-k\\
        \end{align*}
    \item Example 1
        \begin{itemize}
            \item Find average of a set of booleans $f(X) = \frac{1}{n} \sum_{i=1}^n X_i$
            \item $\Delta f = \frac{1}{n}$, add noise $\sim Lap(\frac{1}{n\epsilon})$
            \item Let $X, \Xprime$ differ at $j^{th}$ value,
                \begin{align*}
                    \prob{\Xprime = c}
                \end{align*}
            \item $\prob{}$
        \end{itemize}
    \item Example 1
        \begin{itemize}
            \item How many in the database $\dots$ ?
            \item Sensitivity of the query is 1
            \item Sufficient to add noise $\sim Lap(1/\epsilon)$
            \item \blue{Sampling error $\Omega(\sqrt{n})$ ?}
        \end{itemize}
    \item Example 2 - Histogram
        \begin{itemize}
            \item Most popular first name in a given set of names?
            \item Each person maybe present in at-most one bin
            \item Sensitivity of the histogram query is 1
            \item Sufficient to add noise $\sim Lap(1/\epsilon)$
        \end{itemize}
    \item Example 3 - Vector Valued Queries
        \begin{itemize}
            \item Sufficient to add noise $\sim [Lap(\Delta f/\epsilon)]^d$ where $d$ is the dimension of the independent queries
        \end{itemize}
\end{itemize}

\subsubsection{Exponential Mechanism}

\begin{itemize}
    \item \href{https://en.wikipedia.org/wiki/Exponential_mechanism_(differential_privacy)}{Wikipedia link}
\end{itemize}

\end{document}
